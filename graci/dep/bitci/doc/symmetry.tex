\documentclass[12pt]{article}

\usepackage{amsmath}
\usepackage{amssymb}
\usepackage{titling}
\usepackage[utf8]{inputenc}
\usepackage[T1]{fontenc}
\usepackage{geometry}
\usepackage{array, caption, floatrow, tabularx, makecell, booktabs}%
\captionsetup{labelfont = sc}
\setcellgapes{3pt}

% Font for routines, variables, code
\newcommand*\code[1]{\mbox{\texttt{{#1}}}}

\begin{document}

\predate{}
\postdate{}

\author{}
\title{Symmetry Labels and Indexing}
\date{} % clear date

\maketitle

\section{Point groups}
Shown below are the indices used to label each point group in the
\code{bitCI} library.

\vspace{1cm}

\begin{tabular}{|l|l|}
  \hline
  $\mathbf{C_{1}}$ & 1 \\
  $\mathbf{C_{i}}$ & 2 \\
  $\mathbf{C_{2}}$ & 3 \\
  $\mathbf{C_{s}}$ & 4 \\
  $\mathbf{C_{2h}}$ & 5 \\
  $\mathbf{C_{2v}}$ & 6 \\
  $\mathbf{D_{2}}$  & 7 \\
  $\mathbf{D_{2h}}$ & 8 \\
  \hline
\end{tabular}

\section{Irreducible representations}

\noindent
Tabulated below are the orderings of the irreps of the Abelian point
groups used in the \code{bitCI} library. Also given are the indices,
starting from $0$, used to index each irrep.

\vspace{1cm}

\begin{tabular}{c}
  $\mathbf{C_{1}}$ \\
  \begin{tabular}{|c|c|}
    \hline
    $A$ & $0$ \\
    \hline
  \end{tabular} \\
\end{tabular}

\vspace{1cm}

\begin{tabular}{ccc}
  $\mathbf{C_{2}}$ & $\mathbf{C_{i}}$ & $\mathbf{C_{s}}$ \\
  \begin{tabular}{|c|c|}
    \hline
    $A$ & $0$ \\
    $B$ & $1$ \\
    \hline
  \end{tabular} &
  \begin{tabular}{|c|c|}
    \hline
    $A_{g}$ & $0$ \\
    $A_{u}$ & $1$ \\
    \hline
  \end{tabular} &
  \begin{tabular}{|c|c|}
    \hline
    $A'$ & $0$ \\
    $A''$ & $1$ \\
    \hline
  \end{tabular} \\
\end{tabular}

\vspace{1cm}

\begin{tabular}{ccc}
  $\mathbf{C_{2h}}$ & $\mathbf{C_{2v}}$ & $\mathbf{D_{2}}$ \\
  \begin{tabular}{|c|c|}
    \hline
    $A_{g}$ & $0$ \\
    $B_{g}$ & $1$ \\
    $A_{u}$ & $2$ \\
    $B_{u}$ & $3$ \\
    \hline
  \end{tabular} &
  \begin{tabular}{|c|c|}
    \hline
    $A_{1}$ & $0$ \\
    $A_{2}$ & $1$ \\
    $B_{1}$ & $2$ \\
    $B_{2}$ & $3$ \\
    \hline
  \end{tabular} &
   \begin{tabular}{|c|c|}
    \hline
    $A_{1}$ & $0$ \\
    $B_{1}$ & $1$ \\
    $B_{2}$ & $2$ \\
    $B_{3}$ & $3$ \\
    \hline
  \end{tabular} \\
\end{tabular}

\vspace{1cm}

\begin{tabular}{c}
  $\mathbf{D_{2h}}$ \\
  \begin{tabular}{|c|c|}
    \hline
    $A_{1g}$ & $0$ \\
    $B_{1g}$ & $1$ \\
    $B_{2g}$ & $2$ \\
    $B_{3g}$ & $3$ \\
    $A_{1u}$ & $4$ \\
    $B_{1u}$ & $5$ \\
    $B_{2u}$ & $6$ \\
    $B_{3u}$ & $7$ \\
    \hline
  \end{tabular} \\
\end{tabular}

\end{document}