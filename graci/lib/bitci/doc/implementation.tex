\documentclass[12pt,a4paper]{report}

\usepackage{psfrag}
\usepackage{colordvi}
\usepackage{color}
\usepackage{graphics}
\usepackage{graphicx}
\usepackage{subfigure}
\usepackage{epsf}
\usepackage{nomencl}
\usepackage{multicol}
\usepackage{multirow}
\usepackage{fancyheadings}
\usepackage{rotating}
\usepackage{amsmath}
\usepackage{amsfonts}
\usepackage{cite}
\usepackage{notoccite}
\usepackage{caption}
\usepackage{colortbl}
\usepackage{lscape}
\usepackage{appendix}
\usepackage[normalem]{ulem}
\usepackage{stmaryrd}
\usepackage{soul}
\usepackage{algorithm}
\usepackage{algpseudocode}

\newcommand*\code[1]{\mbox{\texttt{{#1}}}} % routines, variables, code

\begin{document}

%----------------------------------------------------------------------
% Representation of the electronic Hamiltonian
%----------------------------------------------------------------------
\chapter{CSF representation of the electronic Hamiltonian}
\section{CSF expansion of the electronic wavefunctions}
Following Segal and Wetmore\cite{segal_ci_matrix_elements_1975}, let
each CSF $| \text{w} \omega \rangle$ be specifed by a spatial
occupation $\text{w}$ and a spin-coupling $\omega$.

\section{Hamiltonian matrix build}
\subsection{Form of the Hamiltonian}
The second quantisation form of the electronic Hamiltonian reads

\begin{equation}
  \hat{H} = \sum_{ij} h_{ij} \hat{E}_{i}^{j} + \frac{1}{2} \sum_{ijkl}
  V_{ijkl} \left( \hat{E}_{i}^{j} \hat{E}_{k}^{l} + \delta_{jk}
  \hat{E}_{i}^{l} \right),
\end{equation}

\noindent
where $\hat{E}_{i}^{j}$ denotes a spin-traced singlet exciation
operator,

\begin{equation}
  \hat{E}_{i}^{j} = \sum_{\sigma=\alpha,\beta}
  \hat{a}_{i\sigma}^{\dagger} \hat{a}_{j\sigma},
\end{equation}

\noindent
$h_{ij}$ is an element of the core Hamiltonian matrix, and $V_{ijkl}$
is a two-electron integral in the Chemist's notation.

Following Segal and Wetmore\cite{segal_ci_matrix_elements_1975}, the
electronic Hamiltonian may be re-expressed in the following form:

\begin{equation}\label{eq:ham_segal}
  \begin{aligned}
    \hat{H} &= E_{\text{SCF}} - \sum_{i} F_{ii} + \frac{1}{2}
    \sum_{ij} \left(V_{iijj} - \frac{1}{2} V_{ijji} \right)
    \bar{\text{w}}_{i} \bar{\text{w}}_{j} + \sum_{ij} F_{ij}
    \hat{E}_{i}^{j} - \sum_{ijkl} \left( V_{ijkk} - \frac{1}{2}
    V_{ikkj} \right) \bar{\text{w}}_{k} \hat{E}_{i}^{j} \\
    &+ \frac{1}{2} \sum_{ijkl} V_{ijkl} \left( \hat{E}_{i}^{j}
    \hat{E}_{k}^{l} + \delta_{jk} \hat{E}_{i}^{l} \right)
  \end{aligned}
\end{equation}

\noindent
Here, $\bar{\text{w}}$ denotes the spatial occupation of the base
configuration, and $E_{\text{SCF}}$ and $\boldsymbol{F}$ are the SCF
energy and Fock matrix evaluated in the basis of the one-electron
basis being used:

\begin{equation}
  F_{ij} = h_{ij} + \sum_{k} \left(V_{ijkk} - \frac{1}{2} V_{ikkj}
  \right) \bar{\text{w}}_{k},
\end{equation}

\begin{equation}
  E_{\text{SCF}} = \sum_{i} F_{ii} \bar{\text{w}} - \frac{1}{2}
  \sum_{ij} \left( V_{iijj} - \frac{1}{2} V_{ijji} \right)
  \bar{\text{w}}_{i} \bar{\text{w}}_{j}.
\end{equation}

\noindent
Note that $E_{\text{SCF}}$ is not the Hartree-Fock (HF) energy, and
$\boldsymbol{F}$ is not diagonal if non-HF MOs are used as the
single-particle basis.

The form of the electronic Hamiltonian given in Equation
\ref{eq:ham_segal} may at first look redundant and unecessarily
complicated. However, it's use leads to Hamiltonian matrix element
expressions that contain only terms expressed with respect to
singly-occupied MOs in the ket configuration and MOs with different
occupations relative to the base configuration $\bar{\text{w}}$. In
turn, this results in significant computational savings. This is
especially true if the configurations admitted into the CSF basis are
limited to a maximum number of open shells and excitation degree
relative to the base configuration. To understand how this reduction
in terms occurs, it is first necessary to consider some elementary
properties of the singlet excitation operators $\hat{E}_{i}^{j}$.

\subsection{Properties of the singlet excitation operators}
\subsubsection{Commutation relations}
The elementary Fermionic creation and annihilation operators satisfy
the following anti-commutation relations:

\begin{equation}
  [ \hat{a}_{i\sigma}^{\dagger}, \hat{a}_{j\tau}^{\dagger} ]_{+} = [
    \hat{a}_{i\sigma}, \hat{a}_{j\tau} ]_{+} = 0,
\end{equation}

\begin{equation}
  [ \hat{a}_{i\sigma}^{\dagger}, \hat{a}_{j\tau} ]_{+} = \delta_{ij}
  \delta_{\sigma\tau}.
\end{equation}

\noindent
Using these relations, it is found that the singlet excitation
operators obey the following commutation relation:

\begin{equation}\label{eq:generator_commutation}
  [\hat{E}_{i}^{j}, \hat{E}_{k}^{l}] = \delta_{jk} \hat{E}_{i}^{l} -
  \delta_{il} \hat{E}_{k}^{j}.
\end{equation}

\subsubsection{Number operators}\label{sec:number_operators}
The diagonal singlet excitation operator $\hat{E}_{i}^{i}$ corresponds
to a number operator in the sense that taking it's expectation value
with respect to any CSF $| \text{w} \omega \rangle$ yields the
occupation of the $i$th MO in the configuration $\text{w}$:

\begin{equation}\label{eq:number_operator}
  \langle \text{w} \omega | \hat{E}_{i}^{i} | \text{w} \omega \rangle
  = \text{w}_{i}
\end{equation}

This is readily verified by expanding the CSF $| \text{w} \omega
\rangle$ in terms of Slater determinants $| \boldsymbol{k} \rangle$:

\begin{equation}
  | \text{w} \omega \rangle = \sum_{\boldsymbol{K}} C_{\boldsymbol{K}}
  | \boldsymbol{K} \rangle,
\end{equation}

\begin{equation}
    \langle \text{w} \omega | \hat{E}_{i}^{i} | \text{w} \omega
    \rangle = \sum_{\boldsymbol{KL}} C_{\boldsymbol{L}}
    C_{\boldsymbol{K}} \sum_{\sigma} \langle \boldsymbol{L}
    |\hat{a}_{i\sigma}^{\dagger} \hat{a}_{i\sigma} | \boldsymbol{K}
    \rangle = \sum_{\boldsymbol{K}} |C_{\boldsymbol{K}}|^{2}
    \sum_{\sigma} K_{i\sigma}.
\end{equation}

\noindent
Now, the quantity $\sum_{\sigma} K_{i\sigma}$ is independent of the
determinant $| \boldsymbol{K} \rangle$, being just the occupation of
the $i$th spatial MO, i.e., $\sum_{\sigma} K_{i\sigma} =
\text{w}_{i}$. We thus arrive at the desired result:

\begin{equation}
  \langle \text{w} \omega | \hat{E}_{i}^{i} | \text{w} \omega \rangle
  = \text{w}_{i} \sum_{\boldsymbol{K}} |C_{\boldsymbol{K}}|^{2} =
  \text{w}_{i}.
\end{equation}

One last useful result is that the number operator matrix elements
between CSFs with the same spatial configuration but different spin
couplings are zero, i.e.,

\begin{equation}
  \langle \text{w} \omega' | \hat{E}_{i}^{i} | \text{w} \omega \rangle
  = 0.
\end{equation}

\noindent
This is again easily verified by
inserting the Slater determinant expansions of the two CSFs.

%----------------------------------------------------------------------
% Hamiltonian matrix working equations
%----------------------------------------------------------------------
\subsection{Hamiltonian matrix elements}
Using the above properties of the singlet excitation operators and the
form of the electronic Hamiltonian given in Equation
\ref{eq:ham_segal}, we will show how the Hamiltonian matrix elements
reduce to contain only terms involving open shell MOs in the ket
configuration and those MOs differently occupied relative to the base
determinant. The complete derivation is somewhat lengthy and tedious,
and instead we take as our starting point Equations 15-17 of Reference
\citenum{segal_ci_matrix_elements_1975}.

\subsubsection{On-diagonal elements}
In Reference \citenum{segal_ci_matrix_elements_1975}, the on-diagonal
Hamiltonian matrix elements are given as

\begin{equation}\label{eq:hii_segal}
  \begin{aligned}
    \langle \text{w} \omega | \hat{H} - E_{\text{SCF}} | \text{w}
    \omega \rangle &= \sum_{i} F_{ii} \Delta \text{w}_{i} +
    \frac{1}{2} \sum_{ij} \left( V_{iijj} - \frac{1}{2} V_{ijji}
    \right) \Delta \text{w}_{i} \Delta \text{w}_{j} \\
    &+ \frac{1}{2} \sum_{i \ne j} V_{ijji}
    \eta_{ij}^{ji}(\text{w},\omega,\text{w},\omega) + \frac{1}{2}
    \sum_{ij} V_{ijji} \left( \frac{1}{2} \text{w}_{i} \text{w}_{j} -
    \text{w}_{i} \right),
  \end{aligned}
\end{equation}

\noindent
where

\begin{equation}
  \Delta \text{w}_{i} = \text{w}_{i} - \bar{\text{w}}_{i},
\end{equation}

\noindent
and

\begin{equation}\label{eq:eta_ijji}
  \eta_{ij}^{ji}(\text{w},\omega,\text{w},\omega) = \langle \text{w}
  \omega | \hat{E}_{i}^{j} \hat{E}_{j}^{i} | \text{w} \omega \rangle.
\end{equation}

Let $U_{\text{w}}$, $S_{\text{w}}$ and $D_{\text{w}}$ denote the sets
of indices of the MOs that are unoccupied, singly-occupied and
doubly-occupied in the configuration $\text{w}$, respectively. Then,
Equation \ref{eq:hii_segal} can be re-written as

\begin{equation}
  \begin{aligned}
    \langle \text{w} \omega | \hat{H} - E_{\text{SCF}} | \text{w}
    \omega \rangle &= \sum_{i} F_{ii} \Delta \text{w}_{i} +
    \frac{1}{2} \sum_{ij} \left( V_{iijj} - \frac{1}{2} V_{ijji}
    \right) \Delta \text{w}_{i} \Delta \text{w}_{j} \\
    &+ \frac{1}{2} \sum_{i \in S_{\text{w}}} \sum_{j \in S_{\text{w}}}
    V_{ijji} \left[ \eta_{ij}^{ji}(\text{w},\omega,\text{w},\omega)
      (1-\delta_{ij}) + \frac{1}{2} \text{w}_{i} \text{w}_{j} -
      \text{w}_{i} \right] \\
    &+ \frac{1}{2} \sum_{i \in S_{\text{w}}} \sum_{j \in U_{\text{w}}}
    V_{ijji} \left[ \eta_{ij}^{ji}(\text{w},\omega,\text{w},\omega) +
      \frac{1}{2} \text{w}_{i} \text{w}_{j} - \text{w}_{i} \right] \\
    &+ \frac{1}{2} \sum_{i \in D_{\text{w}}} \sum_{j \in S_{\text{w}}}
    V_{ijji} \left[ \eta_{ij}^{ji}(\text{w},\omega,\text{w},\omega) +
      \frac{1}{2} \text{w}_{i} \text{w}_{j} - \text{w}_{i} \right] \\
    &+ \frac{1}{2} \sum_{i \in D_{\text{w}}} \sum_{j \in U_{\text{w}}}
    V_{ijji} \left[ \eta_{ij}^{ji}(\text{w},\omega,\text{w},\omega) +
      \frac{1}{2} \text{w}_{i} \text{w}_{j} - \text{w}_{i} \right]
  \end{aligned}
\end{equation}

Using the commutation relation \ref{eq:generator_commutation} and
Equation \ref{eq:number_operator}, we find that

\begin{equation}
  \eta_{ij}^{ji}(\text{w},\omega,\text{w},\omega) =
  \eta_{ji}^{ij}(\text{w},\omega,\text{w},\omega) + \text{w}_{i} -
  \text{w}_{j},
\end{equation}

\noindent
which allows us to write

\begin{equation}
  \begin{aligned}
    \langle \text{w} \omega | \hat{H} - E_{\text{SCF}} | \text{w}
    \omega \rangle &= \sum_{i} F_{ii} \Delta \text{w}_{i} +
    \frac{1}{2} \sum_{ij} \left( V_{iijj} - \frac{1}{2} V_{ijji}
    \right) \Delta \text{w}_{i} \Delta \text{w}_{j} \\
    &+ \frac{1}{2} \sum_{i \in S_{\text{w}}} \sum_{j \in S_{\text{w}}}
    V_{ijji} \left[ \eta_{ij}^{ji}(\text{w},\omega,\text{w},\omega)
      (1-\delta_{ij}) + \frac{1}{2} \text{w}_{i} \text{w}_{j} -
      \text{w}_{i} \right] \\
    &+ \frac{1}{2} \sum_{i \in S_{\text{w}}} \sum_{j \in U_{\text{w}}}
    V_{ijji} \left[ \eta_{ji}^{ij}(\text{w},\omega,\text{w},\omega) +
      \frac{1}{2} \text{w}_{i} \text{w}_{j} - \text{w}_{j} \right] \\
    &+ \frac{1}{2} \sum_{i \in D_{\text{w}}} \sum_{j \in S_{\text{w}}}
    V_{ijji} \left[ \eta_{ji}^{ij}(\text{w},\omega,\text{w},\omega) +
      \frac{1}{2} \text{w}_{i} \text{w}_{j} - \text{w}_{j} \right] \\
    &+ \frac{1}{2} \sum_{i \in D_{\text{w}}} \sum_{j \in U_{\text{w}}}
    V_{ijji} \left[ \eta_{ji}^{ij}(\text{w},\omega,\text{w},\omega) +
      \frac{1}{2} \text{w}_{i} \text{w}_{j} - \text{w}_{j} \right]
  \end{aligned}
\end{equation}

\noindent
Now, from Equation \ref{eq:eta_ijji} we see that
$\eta_{ji}^{ij}(\text{w},\omega,\text{w},\omega) = 0$ if $j \in
U_{\text{w}}$ or $i \in D_{\text{w}}$. Using this, and inserting the
values of the occupations $\text{w}_{i}$, we arrive at the following
working equation:

\begin{equation}\label{eq:hii_working}
  \begin{aligned}
    \langle \text{w} \omega | \hat{H} - E_{\text{SCF}} | \text{w}
    \omega \rangle &= \sum_{i} F_{ii} \Delta \text{w}_{i} +
    \frac{1}{2} \sum_{ij} \left( V_{iijj} - \frac{1}{2} V_{ijji}
    \right) \Delta \text{w}_{i} \Delta \text{w}_{j} \\
    &+ \frac{1}{2} \sum_{i \in S_{\text{w}}} \sum_{j \in S_{\text{w}}}
    V_{ijji} \left[ \eta_{ij}^{ji}(\text{w},\omega,\text{w},\omega)
      (1-\delta_{ij}) - \frac{1}{2} \right]
  \end{aligned}
\end{equation}

\subsubsection{Off-diagonal elements (i): same spatial occupation, different spin-coupling}
For $\omega \ne \omega'$, we have

\begin{equation}\label{eq:hij_zero_segal}
  \langle \text{w} \omega' | \hat{H} | \text{w} \omega \rangle =
  \frac{1}{2} \sum_{i \ne j} V_{ijji}
  \eta_{ij}^{ji}(\text{w},\omega',\text{w},\omega).
\end{equation}

Decomposing the sum in this equation into individual contributions
from the different excitation classes gives

\begin{equation}
  \begin{aligned}
    \langle \text{w} \omega' | \hat{H} | \text{w} \omega \rangle &=
    \frac{1}{2} \sum_{i \in S_{\text{w}}} \sum_{j \in U_{\text{w}}}
    V_{ijji} \eta_{ij}^{ji}(\text{w},\omega',\text{w},\omega) \\
    &+ \frac{1}{2} \sum_{i \in S_{\text{w}}} \sum_{j \in S_{\text{w}}}
    V_{ijji} \eta_{ij}^{ji}(\text{w},\omega',\text{w},\omega) \\
    &+ \frac{1}{2} \sum_{i \in D_{\text{w}}} \sum_{j \in U_{\text{w}}}
    V_{ijji} \eta_{ij}^{ji}(\text{w},\omega',\text{w},\omega) \\
    &+ \frac{1}{2} \sum_{i \in D_{\text{w}}} \sum_{j \in s_{\text{w}}}
    V_{ijji} \eta_{ij}^{ji}(\text{w},\omega',\text{w},\omega).
  \end{aligned}
\end{equation}

Now, from Equation \ref{eq:generator_commutation} we see that

\begin{equation}
  \begin{aligned}
    \eta_{ij}^{ji}(\text{w},\omega',\text{w},\omega) &= \langle
    \text{w} \omega' | \hat{E}_{j}^{i} \hat{E}_{i}^{j} | \text{w}
    \omega \rangle + \langle \text{w} \omega' | \hat{E}_{i}^{i} |
    \text{w} \omega \rangle - \langle \text{w} \omega' |
    \hat{E}_{j}^{j} | \text{w} \omega \rangle \\
    &= \langle \text{w} \omega' | \hat{E}_{j}^{i} \hat{E}_{i}^{j} |
    \text{w} \omega \rangle.
  \end{aligned}
\end{equation}

\noindent
We thus see that $\eta_{ij}^{ji}(\text{w},\omega',\text{w},\omega)=0$
if $i \in D_{\text{w}}$ or $j \in U_{\text{w}}$. Using this property,
Equation \ref{eq:hij_zero_segal} reduces to the following working
equation:

\begin{equation}\label{eq:hij_zero_working}
  \langle \text{w} \omega' | \hat{H} | \text{w} \omega \rangle =
  \frac{1}{2} \sum_{i \in S_{\text{w}}} \sum_{j \in S_{\text{w}}}
  V_{ijji} \eta_{ij}^{ji}(\text{w},\omega',\text{w},\omega)
  (1-\delta_{ij}).
\end{equation}

\subsubsection{Off-diagonal elements (ii): one-electron differences}
Let the spatial occupations $\text{w}$ and $\text{w}'$ be linked by
the single excitation from MO $i$ to MO $a$. Then,

\begin{equation}\label{eq:hij_single_segal}
  \begin{aligned}
    \langle \text{w}' \omega' | \hat{H} | \text{w} \omega \rangle &=
    \left[ F_{ia} + \sum_{k} \left( V_{iakk} - \frac{1}{2} V_{ikka}
      \right) \Delta \text{w}_{k} \right]
    \eta_{a}^{i}(\text{w}',\omega',\text{w},\omega) \\
    &+ \sum_{k \ne i,a} V_{ikka} \left[
      \eta_{ak}^{ki}(\text{w}',\omega',\text{w},\omega) + \left(
      \frac{1}{2}\text{w}_{k} -1 \right)
      \eta_{a}^{i}(\text{w}',\omega',\text{w},\omega) \right] \\
    &+ \left[ V_{aaai} \text{w}_{a} + V_{aiii} \left( \text{w}_{i} -2
      \right) \right] \eta_{a}^{i}(\text{w}',\omega',\text{w},\omega),
  \end{aligned}
\end{equation}

\noindent
where

\begin{equation}
  \eta_{a}^{i}(\text{w}',\omega',\text{w},\omega) = \langle \text{w}'
  \omega' | \hat{E}_{a}^{i} | \text{w} \omega \rangle.
\end{equation}

\noindent
Now, the sum in Equation \ref{eq:hij_single_segal} involving the
integrals $V_{ikka}$, $k \ne i,a$, contains non-zero contributions
only for $k \in S_{\text{w}}$, which can be shown as follows.

For $k \in D_{\text{w}}$, we have $\text{w}_{k}=2$ and, for $k \ne i$,

\begin{equation}\label{eq:eta_akki_kinD}
  \eta_{ak}^{ki}(\text{w}',\omega',\text{w},\omega) = \langle
  \text{w}' \omega' | \hat{E}_{a}^{k} \hat{E}_{k}^{i} | \text{w}
  \omega \rangle = 0.
\end{equation}

\noindent
We thus see that

\begin{equation}
  \eta_{ak}^{ki}(\text{w}',\omega',\text{w},\omega) + \left(
  \frac{1}{2} \text{w}_{k} -1 \right)
  \eta_{a}^{i}(\text{w}',\omega',\text{w},\omega) = 0, \hspace{0.5cm}
  k \in D_{w}, k \ne i.
\end{equation}

To show that the contibutions for $k \in U_{\text{w}}$ are zero, we
first make use of the commutation relation

\begin{equation}
  [\hat{E}_{a}^{k}, \hat{E}_{k}^{i}] = \hat{E}_{a}^{i}, \hspace{0.5cm}
  i \ne a.
\end{equation}

\noindent
This allows us to write

\begin{equation}
  \begin{aligned}
  \eta_{ak}^{ki}(\text{w}',\omega',\text{w},\omega) &= \langle
  \text{w}' \omega' | \hat{E}_{a}^{k} \hat{E}_{k}^{i} | \text{w}
  \omega \rangle \\
  &= \langle \text{w}' \omega' | \hat{E}_{k}^{i} \hat{E}_{a}^{k} |
  \text{w} \omega \rangle + \langle \text{w}' \omega' |
  \hat{E}_{a}^{i} | \text{w} \omega \rangle \\
  &= \langle \text{w}' \omega' | \hat{E}_{a}^{i} | \text{w} \omega
  \rangle, \hspace{0.5cm} \text{for } \hspace{0.5cm} k \in
  U_{\text{w}}.
  \end{aligned}
\end{equation}

\noindent
That is, for $k \in U_{\text{w}}$,
$\eta_{ak}^{ki}(\text{w}',\omega',\text{w},\omega) =
\eta_{a}^{i}(\text{w}',\omega',\text{w},\omega)$. This allows us to
write the contibutions for $k \in U_{\text{w}}$ as

\begin{equation}
  \eta_{ak}^{ki}(\text{w}',\omega',\text{w},\omega) + \left(
  \frac{1}{2} \text{w}_{k} - 1 \right)
  \eta_{a}^{i}(\text{w}',\omega',\text{w},\omega) = \frac{1}{2}
  \text{w}_{k} \eta_{a}^{i}(\text{w}',\omega',\text{w},\omega) =
  0, \hspace{0.5cm} k \in U_{\text{w}}.
\end{equation}

Collecting together these results, we arrive at the following working
equation:

\begin{equation}\label{eq:hij_single_working}
  \begin{aligned}
   \langle \text{w}' \omega' | \hat{H} | \text{w} \omega \rangle &=
   \left[ F_{ia} + \sum_{k} \left( V_{iakk} - \frac{1}{2} V_{ikka}
     \right) \Delta \text{w}_{k} \right]
   \eta_{a}^{i}(\text{w}',\omega',\text{w},\omega) \\
   &+ \sum_{\substack{k \in S_{\text{w}} \\ k \ne i,a}} V_{ikka}
   \left[ \eta_{ak}^{ki}(\text{w}',\omega',\text{w},\omega) + \left(
     \frac{1}{2}\text{w}_{k} -1 \right)
     \eta_{a}^{i}(\text{w}',\omega',\text{w},\omega) \right] \\
   &+ \left[ V_{aaai} \text{w}_{a} + V_{aiii} \left( \text{w}_{i} -2
     \right) \right] \eta_{a}^{i}(\text{w}',\omega',\text{w},\omega).
  \end{aligned}
\end{equation}

\subsubsection{Off-diagonal elements (iii): two-electron differences}
Let the spatial occupations $\text{w}$ and $\text{w}'$ be linked by
the two excitations from MO $i$ to MO $a$, and from MO $j$ to MO
$b$. Then our working equation simply reads

\begin{equation}\label{eq:hij_double_working}
  \langle \text{w}' \omega' | \hat{H} | \text{w} \omega \rangle =
  \left[ V_{aibj} \eta_{ab}^{ij}(\text{w}',\omega',\text{w},\omega) +
    V_{ajbi} \eta_{ab}^{ji}(\text{w}',\omega',\text{w},\omega) \right]
  \left[ (1+\delta_{ab}) (1+\delta_{ij}) \right]^{-1}.
\end{equation}

%----------------------------------------------------------------------
% (Transition) Density matrix working equations
%----------------------------------------------------------------------
\chapter{(Transition) Density matrices}
\section{One-electron reduced density matrices}
Let the $I$'th electronic state be denoted by

\begin{equation}
  | \Psi_{I} \rangle = \sum_{\text{w}\omega} =
  C_{\text{w}\omega}^{(I)} | \text{w} \omega \rangle.
\end{equation}

The one-electron reduced density matrix (1-RDM) for the state $|
\Psi_{I} \rangle$, $\boldsymbol{\rho}^{(I)}$, has elements

\begin{equation}
  \rho_{pq}^{(I)} = \langle \Psi_{I} | \hat{E}_{p}^{q} | \Psi_{I}
  \rangle.
\end{equation}

By making use of the relations given in Section
\ref{sec:number_operators}, we arrive at the following working
equations for the 1-RDMs:

\begin{equation}
  \rho_{pp}^{(I)} = \sum_{\text{w}\omega} \left|
  C_{\text{w}\omega}^{(I)} \right|^{2} \text{w}_{p},
\end{equation}

\begin{equation}
  \rho_{pq}^{(I)} = \sum_{\substack{\text{w}\ne\text{w}'
      \\ \omega\ne\omega'}} C_{\text{w}'\omega'}^{(I)}
  C_{\text{w}\omega}
  \eta_{p}^{q}(\text{w}',\omega',\text{w},\omega), \hspace{0.5cm}
  p \ne q.
\end{equation}

\section{One-electron transition density matrices}

%----------------------------------------------------------------------
% Bit string representation of configurations
%----------------------------------------------------------------------
\chapter{Bit string representation of spatial configurations}
In order to efficiently evaluate the Hamiltonian matrix working
equations, we need to be able to:

\begin{enumerate}
\item Rapidly compute the excitation degree between any two
  configurations $\text{w}$ and $\text{w}'$ as well as the indices of
  the MOs differently occupied in them.
\item Efficiently retrieve the required spin-coupling coefficients
  $\eta_{i}^{j}(\text{w}',\omega',\text{w},\omega)$ from memory.
\end{enumerate}

\noindent
To achieve this, a dual representation of the configurations
$\text{w}$ was adopted, with one representation tailored to each of
these tasks. These are termed the `configuration' (CONF) and `spatial
occupation pattern' (SOP) representations.

\section{CONF representation}
In the CONF representation, each spatial configuration $\text{w}$ is
represented using a bit string pair $(C_{\text{w}}^{(1)},
C_{\text{w}}^{(2)})$. For $M$ spatial orbitals, $C_{\text{w}}^{(1)}$
and $C_{\text{w}}$ are both composed of $M$ bits. An unnocupied MO
$\phi_{i}$ is represented by unset bits in the $i$th positions of
$C_{\text{w}}^{(1)}$ and $C_{\text{w}}$. A singly-occupied MO
$\phi_{i}$ is represented by a set bits in the $i$th position of
$C_{\text{w}}^{(1)}$ and unset bit in the $i$th position of
$C_{\text{w}}$. A doubly-occupied MO $\phi_{i}$ is represented by a
set bits in both the $i$th position of $C_{\text{w}}^{(1)}$ and
$C_{\text{w}}$. Thus, for example, the spatial configuration

\begin{equation*}
  \text{w} = (2,2,1,0,1,0)
\end{equation*}

\noindent
would be represented by the CONF bit string pair

\begin{equation*}
  \begin{aligned}
    &C_{\text{w}}^{(1)} = 1 1 1 0 1 0 \\
    &C_{\text{w}}^{(2)} = 1 1 0 0 0 0
  \end{aligned}
\end{equation*}

\section{The SOP representation}
In the SOP representation, each spatial configuration $\text{w}$ is
also represented using a bit string pair, denoted by
$(S_{\text{w}}^{(1)}, S_{\text{w}}^{(2)})$. A singly-occupied MO
$\phi_{i}$ is represented by a set bit in $S_{\text{w}}^{(1)}$, and a
doubly-occupied MO $\phi_{i}$ is represented by a set bit in
$S_{\text{w}}^{(2)}$. An unoccupied MO is represented by unset bits in
both $S_{\text{w}}^{(1)}$ and $S_{\text{w}}^{(2)}$. So, the same
configuration

\begin{equation*}
  \text{w} = (2,2,1,0,1,0)
\end{equation*}

\noindent
is represented by the SOP bit string pair

\begin{equation*}
  \begin{aligned}
    &S_{\text{w}}^{(1)} = 0 0 1 0 1 0 \\
    &S_{\text{w}}^{(2)} = 1 1 0 0 0 0
  \end{aligned}
\end{equation*}

\section{Calculation of excitation degrees}
The excitation degree $N_{\text{w}\text{w}'}$ for a pair of
configurations $\text{w}$ and $\text{w}'$ is defined simply as

\begin{equation}
  N_{\text{w}\text{w}'} = \frac{1}{2}\sum_{i=1}^{M} |\text{w}_{i} -
  \text{w}_{i}'|
\end{equation}

\noindent
This quantity may be computed rapidly as follows

\begin{equation}
  N_{\text{w}\text{w}'} = \frac{1}{2} \left( \left|\left|
  C_{\text{w}}^{(1)} \oplus C_{\text{w}'}^{(1)} \right|\right| +
  \left|\left| C_{\text{w}}^{(2)} \oplus C_{\text{w}'}^{(2)}
  \right|\right| \right),
\end{equation}

\noindent
where $\oplus$ denotes the bitwise XOR operation, and $||I||$ is the
Hamming weight of the bit string $I$, that is, the number of set bits
in $I$.

\section{Determination of hole and particle indices}
Given a ket configuration $\text{w}$ and a bra configuration
$\text{w}'$, the hole (particle) MOs are defined as those with
negative (positive) $\text{w}_{i}' - \text{w}_{i}$ values. Using the
CONF representation of $\text{w}$ and $\text{w}'$, the indices of the
hole and particle MOs can be determined as follows.

Firstly, the bit strings

\begin{equation}
  x^{(1)} = C_{\text{w}}^{(1)} \oplus C_{\text{w}'}^{(2)}
\end{equation}

\noindent
and

\begin{equation}
  x^{(2)} = C_{\text{w}}^{(2)} \oplus C_{\text{w}'}^{(2)}
\end{equation}

\noindent
are computed, which have set bits only in the positions of the hole
and particle MOs.

Two bit strings which together encode the positions of the hole MOs
are then computed as

\begin{equation}
  h^{(1)} = C_{\text{w}}^{(1)} \wedge x^{(1)}
\end{equation}

\noindent
and

\begin{equation}
  h^{(2)} = C_{\text{w}}^{(2)} \wedge x^{(2)},
\end{equation}

\noindent
where $\wedge$ denotes the bitwise AND operation. Extracting the
indices of the set bits in $h^{(1)}$ and $h^{(2)}$ then gives the list
of hole indices.

Similarly, the indices of the particle MOs are extracted from the
positions of the set bits in the two bit strings

\begin{equation}
  p^{(1)} = C_{\text{w}'}^{(1)} \wedge x^{(1)}
\end{equation}

\noindent
and

\begin{equation}
  p^{(2)} = C_{\text{w}'}^{(2)} \wedge x^{(2)}.
\end{equation}

\section{Sidenote - extraction of the positions of set bits in a bit string}
Often in the \code{bitci} code, we need to extract the positions of
set bits in some bit string. For example, in the determination of hole
and particle indices. For a give bit string $I$, this is efficiently
accomplished by counting the trailing zeros (unset bits), clearing the
set bits up to the first set bit, and repeating until $I$ is composed
only of unset bits. The pseudo-code for this is given in Algorith
\ref{alg:list_from_bitstring}. Here, $\code{trailz}(I)$ returns the
number of trailing unset bits in the bit string $I$, and
$\code{ibclr}(I,e)$ returns the bit string $I$ with the bit at the
$e$th position cleared (set to zero).

\begin{algorithm}
  \caption{Extraction of the list of set bit indices in a given bit
    string. The input arguments are a bit string $I$ and the array $L$
    to be filled with the positions of the set bits in
    $I$.}\label{alg:list_from_bitstring}

  \begin{algorithmic}[1]

    \Procedure{list\_from\_bitstring}{$I$,$L$}
    
    \State $k \leftarrow 0$ 

    \While{$I \ne 0$}

    \State $e \leftarrow \code{trailz}(I) + 1$
    \State $I \leftarrow \code{ibclr}(I,e)$
    \State $L[k] \leftarrow e$
    \State $k \leftarrow k + 1$
    \EndWhile
    
    \EndProcedure

  \end{algorithmic}
\end{algorithm}

\subsection{Determination of difference configurations}
In the calculation of the Hamiltonian matrix elements, we need to
determine the difference configurations $\Delta \text{w}$, defined as
the vector of differences between the ket and base
configurations. That is,

\begin{equation}
  \Delta \text{w}_i = \text{w}_i - \bar{\text{w}}_i.
\end{equation}

Bit strings encoding of indices of non-zero $\Delta \text{w}_{i}$
values may be computed as follows. First, bit string pairs encoding
the positions of the hole and particle MOs are constructed as

\begin{equation}
  h^{(i)} = \left( C_{\text{w}}^{(i)} \oplus C_{\bar{\text{w}}}^{(i)}
  \right) \wedge C_{\text{w}}^{(i)}, \hspace{0.5cm} i=1,2,
\end{equation}

\noindent
and

\begin{equation}
  p^{(i)} = \left( C_{\text{w}}^{(i)} \oplus C_{\bar{\text{w}}}^{(i)}
  \right) \wedge C_{\bar{\text{w}}}^{(i)}, \hspace{0.5cm} i=1,2,
\end{equation}

\noindent
respectively. Next, separate bit strings encoding the indices of the
MOs with values $\Delta \text{w}_{i}=-2,-1,+1, \text{ and }+2$ are
computed as

\begin{equation}
  d^{(1)} = h^{(1)} \oplus h^{(2)} \hspace{0.5cm} \text{(Positions of
    the MOs for which $\Delta \text{w}_{i} = -1$)},
\end{equation}

\begin{equation}
  d^{(2)} = h^{(1)} \wedge h^{(2)} \hspace{0.5cm} \text{(Positions of
    the MOs for which $\Delta \text{w}_{i} = -2$)},
\end{equation}

\begin{equation}
  a^{(1)} = p^{(1)} \oplus p^{(2)} \hspace{0.5cm} \text{(Positions of
    the MOs for which $\Delta \text{w}_{i} = +1$)},
\end{equation}

\noindent
and

\begin{equation}
  a^{(2)} = p^{(1)} \wedge p^{(2)} \hspace{0.5cm} \text{(Positions of
    the MOs for which $\Delta \text{w}_{i} = +2$)}.
\end{equation}

Finally, the list of non-zero $\Delta \text{w}_{i}$ values and
corresponding MO indices $i$ can be extracted from the bit strings
$d^{(1)}$, $d^{(2)}$, $a^{(1)}$, and $a^{(2)}$ using a slightly
modified version of Algorithm \ref{alg:list_from_bitstring}.

%----------------------------------------------------------------------
% References
%----------------------------------------------------------------------
\newpage
\clearpage
%\bibliographystyle{/home/simon/latex/papers/bib/bst/jcp}
%\bibliography{references}

\clearpage
\lhead{\fancyplain{}{}} \chead{\fancyplain{}{\bfseries Bibliography}}
\cfoot{\fancyplain{\bfseries Bibliography}{\bfseries Bibliography}}
\addcontentsline{toc}{chapter}{Bibliography} \bibliographystyle{./jcp}
\bibliography{./references}

\end{document}

